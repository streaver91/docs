\documentclass[runningheads,letterpaper]{llncs}

\usepackage[utf8]{inputenc}
\usepackage{amssymb}
\setcounter{tocdepth}{3}
\usepackage{graphicx}
\usepackage{url}
\usepackage{upquote}

\begin{document}
\mainmatter
\title{Summary of Git Workflow for Champ Developers}
\titlerunning{Summary of Git Workflow for Champ Developers}

\author{Junhao Li}
\institute{Department of Physics, Cornell University}
\authorrunning{Junhao Li}

\maketitle

\section*{Introduction}
% TODO: Terminology
Vous trouverez des informations pertinentes concernant ce gabarit (en anglais)
dans le reste de ce document.

The \LaTeX{} source of this instruction file for \LaTeX{} users may be
used as a template. This is
located in the ``authors'' subdirectory in
\url{ftp://ftp.springer.de/pub/tex/latex/llncs/latex2e/instruct/} and
entitled \texttt{typeinst.tex}. There is a separate package for Word
users. Kindly send the final and checked source
and PDF files of your paper to the Contact Volume Editor. This is
usually one of the organizers of the conference. You should make sure
that the \LaTeX{} and the PDF files are identical and correct and that
only one version of your paper is sent. It is not possible to update
files at a later stage. Please note that we do not need the printed
paper.



\section{Setup}

You can install git via some basic package-management tool.
If you are on fedora, CentOS, or RedHat, you can use yum
\begin{verbatim}
sudo yum install git-all
\end{verbatim}
If you are on a Debian-based distribution like Ubuntu, you can use apt-get:
\begin{verbatim}
sudo apt-get install git-all
\end{verbatim}
If you are on OS X, you can either download the installer from git website or use Homebrew
\begin{verbatim}
brew install git # You will need to install homebrew first
\end{verbatim}

After installation, set your name and email
\begin{verbatim}
git config --global user.name "Your Name"
git config --global user.email "your.email@example.com"
\end{verbatim}

We host our code on github, so you will also need a github account.
You can sign up for one at \url{github.com} if you do not have.

\section{Basis Branching and Merging}

We illustrate this through an example of fixing a small bug while working on a new feature.

Always get the newest version of the master branch before you start by
\begin{verbatim}
git checkout master  # Switch to master branch
git pull
\end{verbatim}

When working on a new feature, you shall always create a new branch based on master branch and then merge to the master branch when finished.
You can create new branches by
\begin{verbatim}
git checkout master  # Switch to master branch
git branch dev-newfeature # Create a new branch called dev-newfeature
git checkout dev-newfeature  # Switch to that branch
\end{verbatim}
As a convention, we prefix feature branches with ``dev-'' and bug fixing branches with ``fix-''.

Then you can make changes to the files and commit your changes to the new branch.
\begin{verbatim}
vim oldfile1
vim oldfile2
vim newfile
git add newfile  # Make git track the newfile
git status  # If you want to see files changed
git commit -am 'Add xxx'  # Commit your changes
vim oldfile2  # Suppose you fix something wrong in oldfile2
git commit -am 'Fix xxx'  # Commit your changes
...
\end{verbatim}
As a convention,
write the commit as if you are command someone,
commit each small change you make (even one line, and usually smaller than 100 lines),
capitalize the first character,
and always start with ``Fix'' if you are fixing something.

Suppose you while you are working on this new feature, someone reported a bug and you want to deal with it first.
\begin{verbatim}
git checkout master  # Switch back to master
git pull  # Obtain newest master from remote
git branch fix-somebug # Create a new branch based on master
git checkout fix-somebug  # Switch to that branch
vim oldfile3
git commit -am 'Fix xxx' # Commit your changes
...
# Now you finished and ready to merge the fix
git checkout master  # Switch back to master again
git pull
git merge fix-somebug  # Merge fix to master
\end{verbatim}

Then you can continue working on your new feature.
And when you finish, merge it to the master branch.
\begin{verbatim}
vim oldfile4
git commit -am 'Change xxx'  # Commit your changes
...
# Now you finished the new feature
git checkout master  # Switch back to master again
git pull
git merge dev-newfeature  # Merge fix to master
\end{verbatim}

All the commits are local until you push them explicitly.
To push to the remote,
\begin{verbatim}
git push origin some-branch
\end{verbatim}
It will ask for username and password.
You can save your credential information by
\begin{verbatim}
git config credential.helper 'store'
\end{verbatim}

\section{Core Developers Workflow}

Core developers will have the permission to push directly into our git repo.

When start working on a new feature or fix a bug, you first want to checkout the master branch and do a pull to get the latest changes.
\begin{verbatim}
git

\end{verbatim}


\section{Première section}

Vous trouverez des informations pertinentes concernant ce gabarit (en anglais)
dans le reste de ce document.

\end{document}
